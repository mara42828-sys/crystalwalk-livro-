\chapter{Guia do Professor: Como usar este livro com CrystalWalk}

\section*{Resumo}
Este capítulo explica o objetivo do livro-curso, sugestões de cronograma, níveis (básico/intermediário/avançado) e recomendações para avaliar impacto (pré/pós testes, rubricas).

\begin{lessonplan}
\textbf{Objetivos gerais:}
\begin{itemize}
  \item Introduzir conceitos fundamentais de cristalografia de forma prática.
  \item Ensinar o uso do software CrystalWalk para construir e visualizar estruturas.
  \item Fornecer atividades e quizzes para avaliar a aprendizagem.
\end{itemize}

\textbf{Sugestão de cronograma (semestre):}
\begin{itemize}
  \item Módulo 1 (4 semanas): Conceitos básicos + CrystalWalk — 4 aulas.
  \item Módulo 2 (6 semanas): Células unitárias e índices de Miller — 6 aulas.
  \item Módulo 3 (4 semanas): Projetos práticos e avaliação final — 4 aulas.
\end{itemize}
\end{lessonplan}

\section*{Estrutura das aulas com CrystalWalk (passo a passo)}
\begin{enumerate}
  \item Preparação do ambiente (computadores, navegador, acesso ao CrystalWalk).
  \item Aula demonstrativa: abrir o CrystalWalk, exibir tutorial passo a passo do próprio software.
  \item Aula prática guiada: cada estudante deve seguir um roteiro com passos (veja exemplos em cada capítulo).
  \item Avaliação formativa com quizzes ao final de cada aula.
\end{enumerate}
